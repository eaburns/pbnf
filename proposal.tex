\documentclass{article}
\usepackage{fullpage}
\usepackage{algorithmic}

\title{\textbf{Parallel Heuristic Search}}

\author{Ethan Burns and Seth Lemons \\
  Department of Computer Science \\
  University of New Hampshire \\
  Durham, NH 03820 \\
  \{eaburns,seth.lemons\}@unh.edu}

\date{\today}

\begin{document}
\maketitle

\section{Problem}

Over the past few decades we have seen major increases in the
processing speeds of CPUs.  We are coming to a point, however, where
it is no longer feasible to create processors with higher clock
speeds.  It is widely accepted that the next generations of computers
are going to rely on parallelism instead of processor cycles for
increased performance.  In order to make better use of this upcoming
hardware, the artificial intelligence community will need to begin
creating algorithms that make use of concurrency between multiple CPU
cores.

Heuristic search is one of the most fundamental tools used in the
field of artificial intelligence.  This technique is used to explore
possible actions that can be taken in order to achieve a goal and generalizes across many problem domains.  Today, most
heuristic search algorithms are sequential; they only make use of a
single processor core.  In order to adapt the field of artificial
intelligence to the next generation of computers, we believe that it
will be necessary to reformulate search algorithms so that they can
take advantage of parallel computing.

\section{Goal}

Our goal for this project is to explore novel methods of performing
heuristic search on machines with many CPU cores. We hope to examine
algorithms that minimize memory contention while spreading the work
out among cores as efficiently as possible. Any algorithms we design
or examine should be generalizable across as many domains as possible
and efficient regardless of the number of cores available.

\section{Previous Work}

\begin{itemize}

\item
\emph{Parallel Heuristic Search: Two Approaches} -- Curt Powley, Chris
Ferguson and Richard E. Korf.

\item
\emph{Adaptive Parallel Iterative Deepening Search} -- Diane J. Cook
and R. Craig Varnell

\item
\emph{PRA*: Masively Parallel Heuristic Search} -- Matthew Evett,
James Hendler, Ambuj Mahanti and Dana Nau

\item
\emph{A Simple Parallel Algorithm for the Single-Source Shortest Path
  Problem on Planar Digraphs} -- Jesper L. Traeff and Christos
D. Zaroliagis

\item
\emph{Efficient Parallel Algorithms for Computing All Pair Shortest
  Paths in Directed Graphs} -- Yijie Han, V.Y. Pan and J.H. Reif

\item
\emph{Parallel Structured Duplicate Detection} -- Rong Zhou and Eric
A. Hansen

\end{itemize}

\section{Approach}

\begin{itemize}


\item Implement previous algorithms:
  \begin{itemize}
  \item DTS
  \item Parallel window search
  \item PRA*
  \item Parallel A* with shared OPEN list?
  \end{itemize}

\item Create a new parallel best first search algorithm:\\
  \begin{itemize}
  \item search(init, cores)
  \begin{algorithmic}[1]
  \STATE create a thread for each core, giving each a link to its ''right'' neighbor and a queue for messages from its ''left'' neighbor
  \STATE put initial node on first thread's ''left'' queue
  \STATE start threadsearch() on each thread
  \end{algorithmic}

  \item threadsearch()
  \begin{algorithmic}[1]
  \STATE current $\leftarrow$ NULL
  \LOOP
    \FOR {$i=0$ to min(n, length($open$))}
      \STATE $node \leftarrow$ head of $open$
      \IF{$node$ is a goal}
        \STATE create GOAL token
        \STATE gotgoal($node$)
      \ELSE
        \STATE insert children($node$) into $open$
      \ENDIF
    \ENDFOR
    \IF{length($open$) $> 0$}
      \STATE pass head of $open$ ''right''
    \ELSE
      \STATE pass NO WORK token ''right''
    \ENDIF
    \STATE receive $message$ from ''left'' queue
    \IF{$message$ is GOAL token}
      \IF{$message$ originated at this thread and is not marked incumbent}
        \STATE pass TERMINATION token ''right'' then break loop
      \ELSE
        \STATE gotgoal(token)
      \ENDIF
    \ELSIF{$message$ is a node}
      \STATE insert $message$ into $open$
    \ELSIF{$message$ is NO WORK token}
      \IF{$message$ originated at this thread}
        \STATE pass TERMINATION token ''right'' then break loop
      \ELSIF{length($open$) $= 0$}
        \STATE pass token ''right''
      \ENDIF
    \ELSIF{$message$ is TERMINATION token}
      \STATE pass $message$ ''right'' then break loop
    \ENDIF
  \ENDLOOP
  \STATE return $current$
  \end{algorithmic}

  \item gotgoal($new$):
  \begin{algorithmic}[1]
      \IF{g($new$) $<$ g($current$)}
        \STATE prune all nodes off $open$ with g values higher than g($current$)
        \STATE $current \leftarrow new$
        \IF{$open$ is not empty}
          \STATE mark $new$ as incumbent
          \ENDIF
        \STATE pass $new$ ''right''
      \ENDIF
  \end{algorithmic}
  \end{itemize}

\item Refine the new algorithm.

\item Model our new idea using TLA for proofs.

\item Test our new algorithm against previous parallel algorithms
  listed above.

\end{itemize}

\section{Division of Labor}

\begin{itemize}
\item Ethan is in charge of:
  \begin{itemize}
  \item Get numbers from AWA* (which is similar to our algo.)
  \item TLA modeling
  \item Implementation of PRA*
  \item Implementation of Parallel A* with shared OPEN list?
  \item Refinement and implementation of our new approach
  \end{itemize}
\item Seth is in charge of:
  \begin{itemize}
  \item Write paper
  \item Find a multi-core computer to run experiments on
  \item Implementation of DTS
  \item Implementation of Parallel window search
  \item Refinement and implementation of our new approach
  \end{itemize}
\end{itemize}

\section{To Do}

\begin{itemize}
\item Decide on a programming language to use
\item Find a multi-core computer to run experiments on
\item Get numbers from AWA* (which is similar to our algo.)
\item Impl. PRA*
\item Impl. Parallel A* with shared OPEN list?
\item Impl. DTS
\item Impl. Parallel Window Search
\item Refine new approach:
  \begin{itemize}
  \item When to expand and when to pass nodes
  \item CLOSED list?
  \item More?
  \end{itemize}
\item Write paper
\item TLA modeling
\end{itemize}

\end{document}
