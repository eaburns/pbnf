\batchmode %% Suppresses most terminal output.
\documentclass{article}
\usepackage{color}
\definecolor{boxshade}{gray}{0.85}
\setlength{\textwidth}{550pt}
\setlength{\textheight}{725pt}
\addtolength{\hoffset}{-100pt}
\addtolength{\voffset}{-100pt}
\usepackage{latexsym}
\usepackage{ifthen}
% \usepackage{color}
%%%%%%%%%%%%%%%%%%%%%%%%%%%%%%%%%%%%%%%%%%%%%%%%%%%%%%%%%%%%%%%%%%%%%%%%%%%%%
% SWITCHES                                                                  %
%%%%%%%%%%%%%%%%%%%%%%%%%%%%%%%%%%%%%%%%%%%%%%%%%%%%%%%%%%%%%%%%%%%%%%%%%%%%%
\newboolean{shading} 
\setboolean{shading}{false}
\makeatletter
 %% this is needed only when inserted into the file, not when
 %% used as a package file.
%%%%%%%%%%%%%%%%%%%%%%%%%%%%%%%%%%%%%%%%%%%%%%%%%%%%%%%%%%%%%%%%%%%%%%%%%%%%%
%                                                                           %
% DEFINITIONS OF SYMBOL-PRODUCING COMMANDS                                  %
%                                                                           %
%    TLA+      LaTeX                                                        %
%    symbol    command                                                      %
%    ------    -------                                                      %
%    =>        \implies                                                     %
%    <:        \ltcolon                                                     %
%    :>        \colongt                                                     %
%    ==        \defeq                                                       %
%    ..        \dotdot                                                      %
%    ::        \coloncolon                                                  %
%    =|        \eqdash                                                      %
%    ++        \pp                                                          %
%    --        \mm                                                          %
%    **        \stst                                                        %
%    //        \slsl                                                        %
%    ^         \ct                                                          %
%    \A        \A                                                           %
%    \E        \E                                                           %
%    \AA       \AA                                                          %
%    \EE       \EE                                                          %
%%%%%%%%%%%%%%%%%%%%%%%%%%%%%%%%%%%%%%%%%%%%%%%%%%%%%%%%%%%%%%%%%%%%%%%%%%%%%
\newlength{\symlength}
\newcommand{\implies}{\Rightarrow}
\newcommand{\ltcolon}{\mathrel{<\!\!\mbox{:}}}
\newcommand{\colongt}{\mathrel{\!\mbox{:}\!\!>}}
\newcommand{\defeq}{\;\mathrel{\smash   %% keep this symbol from being too tall
    {{\stackrel{\scriptscriptstyle\Delta}{=}}}}\;}
\newcommand{\dotdot}{\mathrel{\ldotp\ldotp}}
\newcommand{\coloncolon}{\mathrel{::\;}}
\newcommand{\eqdash}{\mathrel = \joinrel \hspace{-.28em}|}
\newcommand{\pp}{\mathbin{++}}
\newcommand{\mm}{\mathbin{--}}
\newcommand{\stst}{*\!*}
\newcommand{\slsl}{/\!/}
\newcommand{\ct}{\hat{\hspace{.4em}}}
\newcommand{\A}{\forall}
\newcommand{\E}{\exists}
\renewcommand{\AA}{\makebox{$\raisebox{.05em}{\makebox[0pt][l]{%
   $\forall\hspace{-.517em}\forall\hspace{-.517em}\forall$}}%
   \forall\hspace{-.517em}\forall \hspace{-.517em}\forall\,$}}
\newcommand{\EE}{\makebox{$\raisebox{.05em}{\makebox[0pt][l]{%
   $\exists\hspace{-.517em}\exists\hspace{-.517em}\exists$}}%
   \exists\hspace{-.517em}\exists\hspace{-.517em}\exists\,$}}
\newcommand{\whileop}{\.{\stackrel
  {\mbox{\raisebox{-.3em}[0pt][0pt]{$\scriptscriptstyle+\;\,$}}}%
  {-\hspace{-.16em}\triangleright}}}

% Commands are defined to produce the upper-case keywords.
% Note that some have space after them.
\newcommand{\ASSUME}{\textsc{assume }}
\newcommand{\ASSUMPTION}{\textsc{assumption }}
\newcommand{\AXIOM}{\textsc{axiom }}
\newcommand{\BOOLEAN}{\textsc{boolean }}
\newcommand{\CASE}{\textsc{case }}
\newcommand{\CONSTANT}{\textsc{constant }}
\newcommand{\CONSTANTS}{\textsc{constants }}
\newcommand{\ELSE}{\settowidth{\symlength}{\THEN}%
   \makebox[\symlength][l]{\textsc{ else}}}
\newcommand{\EXCEPT}{\textsc{ except }}
\newcommand{\EXTENDS}{\textsc{extends }}
\newcommand{\FALSE}{\textsc{false}}
\newcommand{\IF}{\textsc{if }}
\newcommand{\IN}{\settowidth{\symlength}{\LET}%
   \makebox[\symlength][l]{\textsc{in}}}
\newcommand{\INSTANCE}{\textsc{instance }}
\newcommand{\LET}{\textsc{let }}
\newcommand{\LOCAL}{\textsc{local }}
\newcommand{\MODULE}{\textsc{module }}
\newcommand{\OTHER}{\textsc{other }}
\newcommand{\STRING}{\textsc{string}}
\newcommand{\THEN}{\textsc{ then }}
\newcommand{\THEOREM}{\textsc{theorem }}
\newcommand{\TRUE}{\textsc{true}}
\newcommand{\VARIABLE}{\textsc{variable }}
\newcommand{\VARIABLES}{\textsc{variables }}
\newcommand{\WITH}{\textsc{ with }}
\newcommand{\WF}{\textrm{WF}}
\newcommand{\SF}{\textrm{SF}}
\newcommand{\CHOOSE}{\textsc{choose }}
\newcommand{\ENABLED}{\textsc{enabled }}
\newcommand{\UNCHANGED}{\textsc{unchanged }}
\newcommand{\SUBSET}{\textsc{subset }}
\newcommand{\UNION}{\textsc{union }}
\newcommand{\DOMAIN}{\textsc{domain }}
% Added for tla2tex
\newcommand{\BY}{\textsc{by }}
\newcommand{\OBVIOUS}{\textsc{obvious }}
\newcommand{\HAVE}{\textsc{have }}
\newcommand{\QED}{\textsc{qed }}
\newcommand{\TAKE}{\textsc{take }}
\newcommand{\DEF}{\textsc{def }}
\newcommand{\HIDE}{\textsc{hide }}
\newcommand{\RECURSIVE}{\textsc{recursive }}
\newcommand{\USE}{\textsc{use }}
\newcommand{\DEFINE}{\textsc{define }}
\newcommand{\PROOF}{\textsc{proof }}
\newcommand{\WITNESS}{\textsc{witness }}
\newcommand{\PICK}{\textsc{pick }}
\newcommand{\DEFS}{\textsc{defs }}
\newcommand{\PROVE}{\textsc{prove }}
\newcommand{\SUFFICES}{\textsc{suffices }}
\newcommand{\NEW}{\textsc{new }}
\newcommand{\LAMBDA}{\textsc{lambda }}
\newcommand{\STATE}{\textsc{state }}
\newcommand{\ACTION}{\textsc{action }}
\newcommand{\TEMPORAL}{\textsc{temporal }}
\newcommand{\@pfstepnum}[2]{\ensuremath{\langle#1\rangle}\textrm{#2}}
\newcommand{\bang}{\@s{1}\mbox{\small !}\@s{1}}

%%%%%%%%%%%%%%%%%%%%%%%%%%%%%%%%%%%%%%%%%%%%%%%%%%%%%%%%%
% REDEFINE STANDARD COMMANDS TO MAKE THEM FORMAT BETTER %
%                                                       %
% We redefine \in and \notin                            %
%%%%%%%%%%%%%%%%%%%%%%%%%%%%%%%%%%%%%%%%%%%%%%%%%%%%%%%%%
\renewcommand{\_}{\rule{.4em}{.06em}\hspace{.05em}}
\newlength{\equalswidth}
\let\oldin=\in
\let\oldnotin=\notin
\renewcommand{\in}{%
   {\settowidth{\equalswidth}{$\.{=}$}\makebox[\equalswidth][c]{$\oldin$}}}
\renewcommand{\notin}{%
   {\settowidth{\equalswidth}{$\.{=}$}\makebox[\equalswidth]{$\oldnotin$}}}


%%%%%%%%%%%%%%%%%%%%%%%%%%%%%%%%%%%%%%%%%%%%%%%%%%%%
%                                                  %
% HORIZONTAL BARS:                                 %
%                                                  %
%   \moduleLeftDash    |~~~~~~~~~~                 %
%   \moduleRightDash    ~~~~~~~~~~|                %
%   \midbar            |----------|                %
%   \bottombar         |__________|                %
%%%%%%%%%%%%%%%%%%%%%%%%%%%%%%%%%%%%%%%%%%%%%%%%%%%%
\newlength{\charwidth}\settowidth{\charwidth}{{\small\tt M}}
\newlength{\boxrulewd}\setlength{\boxrulewd}{.4pt}
\newlength{\boxlineht}\setlength{\boxlineht}{.5\baselineskip}
\newcommand{\boxsep}{\charwidth}
\newlength{\boxruleht}\setlength{\boxruleht}{.5ex}
\newlength{\boxruledp}\setlength{\boxruledp}{-\boxruleht}
\addtolength{\boxruledp}{\boxrulewd}
\newcommand{\boxrule}{\leaders\hrule height \boxruleht depth \boxruledp
                      \hfill\mbox{}}
\newcommand{\@computerule}{%
  \setlength{\boxruleht}{.5ex}%
  \setlength{\boxruledp}{-\boxruleht}%
  \addtolength{\boxruledp}{\boxrulewd}}

\newcommand{\bottombar}{\hspace{-\boxsep}%
  \raisebox{-\boxrulewd}[0pt][0pt]{\rule[.5ex]{\boxrulewd}{\boxlineht}}%
  \boxrule
  \raisebox{-\boxrulewd}[0pt][0pt]{%
      \rule[.5ex]{\boxrulewd}{\boxlineht}}\hspace{-\boxsep}\vspace{0pt}}

\newcommand{\moduleLeftDash}%
   {\hspace*{-\boxsep}%
     \raisebox{-\boxlineht}[0pt][0pt]{\rule[.5ex]{\boxrulewd
               }{\boxlineht}}%
    \boxrule\hspace*{.4em }}

\newcommand{\moduleRightDash}%
    {\hspace*{.4em}\boxrule
    \raisebox{-\boxlineht}[0pt][0pt]{\rule[.5ex]{\boxrulewd
               }{\boxlineht}}\hspace{-\boxsep}}%\vspace{.2em}

\newcommand{\midbar}{\hspace{-\boxsep}\raisebox{-.5\boxlineht}[0pt][0pt]{%
   \rule[.5ex]{\boxrulewd}{\boxlineht}}\boxrule\raisebox{-.5\boxlineht%
   }[0pt][0pt]{\rule[.5ex]{\boxrulewd}{\boxlineht}}\hspace{-\boxsep}}

%%%%%%%%%%%%%%%%%%%%%%%%%%%%%%%%%%%%%%%%%%%%%%%%%%%%%%%%%%%%%%%%%%%%%%%%%%%%%
% FORMATING COMMANDS                                                        %
%%%%%%%%%%%%%%%%%%%%%%%%%%%%%%%%%%%%%%%%%%%%%%%%%%%%%%%%%%%%%%%%%%%%%%%%%%%%%


%\tstrut: A strut to produce inter-paragraph space in a comment.
%\rstrut: A strut to extend the bottom of a one-line comment so
%         there's no break in the shading between comments on 
%         successive lines.
\newcommand\tstrut%
  {\raisebox{\vshadelen}{\raisebox{-.25em}{\rule{0pt}{1.15em}}}%
   \global\setlength{\vshadelen}{0pt}}
\newcommand\rstrut{\raisebox{-.25em}{\rule{0pt}{1.15em}}%
 \global\setlength{\vshadelen}{0pt}}


% \.{op} formats operator op in math mode with empty boxes on either side.
% Used because TeX otherwise vary the amount of space it leaves around op.
\renewcommand{\.}[1]{\ensuremath{\mbox{}#1\mbox{}}}

% \@s{n} produces an n-point space
\newcommand{\@s}[1]{\hspace{#1pt}}           

% \@x{txt} starts a specification line in the beginning with txt
% in the final LaTeX source.
\newcommand{\@x}[1]{\par\mbox{$\mbox{}#1\mbox{}$}}  

% \@xx{txt} continues a specification line with the text txt.
\newcommand{\@xx}[1]{\mbox{$\mbox{}#1\mbox{}$}}  

% \@y{cmt} produces a one-line comment.
\newcommand{\@y}[1]{\mbox{\footnotesize\hspace{.65em}%
  \ifthenelse{\boolean{shading}}{%
      \shadebox{#1\hspace{-\the\lastskip}\rstrut}}%
               {#1\hspace{-\the\lastskip}\rstrut}}}

% \@z{cmt} produces a zero-width one-line comment.
\newcommand{\@z}[1]{\makebox[0pt][l]{\footnotesize
  \ifthenelse{\boolean{shading}}{%
      \shadebox{#1\hspace{-\the\lastskip}\rstrut}}%
               {#1\hspace{-\the\lastskip}\rstrut}}}


% \@w{str} produces the TLA+ string "str".
\newcommand{\@w}[1]{\textsf{``{#1}''}}             


%%%%%%%%%%%%%%%%%%%%%%%%%%%%%%%%%%%%%%%%%%%%%%%%%%%%%%%%%%%%%%%%%%%%%%%%%%%%%
% SHADING                                                                   %
%%%%%%%%%%%%%%%%%%%%%%%%%%%%%%%%%%%%%%%%%%%%%%%%%%%%%%%%%%%%%%%%%%%%%%%%%%%%%
\def\graymargin{1}
  % The number of points of margin in the shaded box.

% \definecolor{boxshade}{gray}{.85}
% Defines the darkness of the shading: 1 = white, 0 = black
% Added by TLATeX only if needed.

% \shadebox{txt} puts txt in a shaded box.
\newlength{\templena}
\newlength{\templenb}
\newsavebox{\tempboxa}
\newcommand{\shadebox}[1]{{\setlength{\fboxsep}{\graymargin pt}%
     \savebox{\tempboxa}{#1}%
     \settoheight{\templena}{\usebox{\tempboxa}}%
     \settodepth{\templenb}{\usebox{\tempboxa}}%
     \hspace*{-\fboxsep}\raisebox{0pt}[\templena][\templenb]%
        {\colorbox{boxshade}{\usebox{\tempboxa}}}\hspace*{-\fboxsep}}}

% \vshade{n} makes an n-point inter-paragraph space, with
%  shading if the `shading' flag is true.
\newlength{\vshadelen}
\setlength{\vshadelen}{0pt}
\newcommand{\vshade}[1]{\ifthenelse{\boolean{shading}}%
   {\global\setlength{\vshadelen}{#1pt}}%
   {\vspace{#1pt}}}

\newlength{\boxwidth}
\newlength{\multicommentdepth}

%%%%%%%%%%%%%%%%%%%%%%%%%%%%%%%%%%%%%%%%%%%%%%%%%%%%%%%%%%%%%%%%%%%%%%%%%%%%%
% THE cpar ENVIRONMENT                                                      %
% ^^^^^^^^^^^^^^^^^^^^                                                      %
% The LaTeX input                                                           %
%                                                                           %
%   \begin{cpar}{pop}{nest}{isLabel}{d}{e}{arg6}                            %
%     XXXXXXXXXXXXXXX                                                       %
%     XXXXXXXXXXXXXXX                                                       %
%     XXXXXXXXXXXXXXX                                                       %
%   \end{cpar}                                                              %
%                                                                           %
% produces one of two possible results.  If isLabel is the letter "T",      %
% it produces the following, where [label] is the result of typesetting     %
% arg6 in an LR box, and d is is a number representing a distance in        %
% points.                                                                   %
%                                                                           %
%   prevailing |<-- d -->[label]<- e ->XXXXXXXXXXXXXXX                      %
%         left |                       XXXXXXXXXXXXXXX                      %
%       margin |                       XXXXXXXXXXXXXXX                      %
%                                                                           %
% If isLabel is the letter "F", then it produces                            %
%                                                                           %
%   prevailing |<-- d -->XXXXXXXXXXXXXXXXXXXXXXX                            %
%         left |         <- e ->XXXXXXXXXXXXXXXX                            %
%       margin |                XXXXXXXXXXXXXXXX                            %
%                                                                           %
% where d and e are numbers representing distances in points.               %
%                                                                           %
% The prevailing left margin is the one in effect before the most recent    %
% pop (argument 1) cpar environments with "T" as the nest argument, where   %
% pop is a number \geq 0.                                                   %
%                                                                           %
% If the nest argument is the letter "T", then the prevailing left          %
% margin is moved to the left of the second (and following) lines of        %
% X's.  Otherwise, the prevailing left margin is left unchanged.            %
%                                                                           %
% An \unnest{n} command moves the prevailing left margin to where it was    %
% before the most recent n cpar environments with "T" as the nesting        %
% argument.                                                                 %
%                                                                           %
% The environment leaves no vertical space above or below it, or between    %
% its paragraphs.  (TLATeX inserts the proper amount of vertical space.)    %
%%%%%%%%%%%%%%%%%%%%%%%%%%%%%%%%%%%%%%%%%%%%%%%%%%%%%%%%%%%%%%%%%%%%%%%%%%%%%

\newcounter{pardepth}
\setcounter{pardepth}{0}

% \setgmargin{txt} defines \gmarginN to be txt, where N is \roman{pardepth}.
% \thegmargin equals \gmarginN, where N is \roman{pardepth}.
\newcommand{\setgmargin}[1]{%
  \expandafter\xdef\csname gmargin\roman{pardepth}\endcsname{#1}}
\newcommand{\thegmargin}{\csname gmargin\roman{pardepth}\endcsname}
\newcommand{\gmargin}{0pt}

\newsavebox{\tempsbox}

\newenvironment{cpar}[6]{%
  \addtocounter{pardepth}{-#1}%
  \ifthenelse{\boolean{shading}}{\par\begin{lrbox}{\tempsbox}%
                                 \begin{minipage}[t]{\linewidth}}{}%
  \begin{list}{}{%
     \edef\temp{\thegmargin}
     \ifthenelse{\equal{#3}{T}}%
       {\settowidth{\leftmargin}{\hspace{\temp}\footnotesize #6\hspace{#5pt}}%
        \addtolength{\leftmargin}{#4pt}}%
       {\setlength{\leftmargin}{#4pt}%
        \addtolength{\leftmargin}{#5pt}%
        \addtolength{\leftmargin}{\temp}%
        \setlength{\itemindent}{-#5pt}}%
      \ifthenelse{\equal{#2}{T}}{\addtocounter{pardepth}{1}%
                                 \setgmargin{\the\leftmargin}}{}%
      \setlength{\labelwidth}{0pt}%
      \setlength{\labelsep}{0pt}%
      \setlength{\itemindent}{-\leftmargin}%
      \setlength{\topsep}{0pt}%
      \setlength{\parsep}{0pt}%
      \setlength{\partopsep}{0pt}%
      \setlength{\parskip}{0pt}%
      \setlength{\itemsep}{0pt}
      \setlength{\itemindent}{#4pt}%
      \addtolength{\itemindent}{-\leftmargin}}%
   \ifthenelse{\equal{#3}{T}}%
      {\item[\tstrut\footnotesize \hspace{\temp}{#6}\hspace{#5pt}]
        }%
      {\item[\tstrut\hspace{\temp}]%
         }%
   \footnotesize}
 {\hspace{-\the\lastskip}\tstrut
 \end{list}%
  \ifthenelse{\boolean{shading}}{\end{minipage}  
                                 \end{lrbox}%
                                 \shadebox{\usebox{\tempsbox}}\par}{}%
  }

%%%%%%%%%%%%%%%%%%%%%%%%%%%%%%%%%%%%%%%%%%%%%%%%%%%%%%%%%%%%%%%%%%%%%%%%%%%%%%
% THE lcom ENVIRONMENT                                                       %
% ^^^^^^^^^^^^^^^^^^^^                                                       %
% A multi-line comment with no text to its left is typeset in an lcom        % 
% environment, whose argument is a number representing the indentation       % 
% of the left margin, in points.  All the text of the comment should be      % 
% inside cpar environments.                                                  % 
%%%%%%%%%%%%%%%%%%%%%%%%%%%%%%%%%%%%%%%%%%%%%%%%%%%%%%%%%%%%%%%%%%%%%%%%%%%%%%
\newenvironment{lcom}[1]{%
  \par\vspace{.2em}%
  \sloppypar
  \setcounter{pardepth}{0}%
  \footnotesize
  \begin{list}{}{%
    \setlength{\leftmargin}{#1pt}
    \setlength{\labelwidth}{0pt}%
    \setlength{\labelsep}{0pt}%
    \setlength{\itemindent}{0pt}%
    \setlength{\topsep}{0pt}%
    \setlength{\parsep}{0pt}%
    \setlength{\partopsep}{0pt}%
    \setlength{\parskip}{0pt}}
    \item[]}%
  {\end{list}\vspace{.3em}%
 }


%%%%%%%%%%%%%%%%%%%%%%%%%%%%%%%%%%%%%%%%%%%%%%%%%%%%%%%%%%%%%%%%%%%%%%%%%%%%%
% THE mcom ENVIRONMENT AND \mutivspace COMMAND                              %
% ^^^^^^^^^^^^^^^^^^^^^^^^^^^^^^^^^^^^^^^^^^^^                              %
%                                                                           %
% A part of the spec containing a right-comment of the form                 %
%                                                                           %
%      xxxx (*************)                                                 %
%      yyyy (* ccccccccc *)                                                 %
%      ...  (* ccccccccc *)                                                 %
%           (* ccccccccc *)                                                 %
%           (* ccccccccc *)                                                 %
%           (*************)                                                 %
%                                                                           %
% is typeset by                                                             %
%                                                                           %
%     XXXX \begin{mcom}{d}                                                  %
%            CCCC ... CCC                                                   %
%          \end{mcom}                                                       %
%     YYYY ...                                                              %
%     \multivspace{n}                                                       %
%                                                                           %
% where the number d is the width in points of the comment, n is the        %
% number of xxxx, yyyy, ...  lines to the left of the comment.              %
% All the text of the comment should be typeset in cpar environments.       %
%                                                                           %
% This puts the comment into a single box (so no page breaks can occur      %
% within it).  The entire box is shaded iff the shading flag is true.       %
%%%%%%%%%%%%%%%%%%%%%%%%%%%%%%%%%%%%%%%%%%%%%%%%%%%%%%%%%%%%%%%%%%%%%%%%%%%%%
\newenvironment{mcom}[1]{%
  \setcounter{pardepth}{0}%
  \hspace{.65em}%
  \begin{lrbox}{\alignbox}\sloppypar%
      \setboolean{shading}{false}%
      \setlength{\boxwidth}{#1pt}%
      \addtolength{\boxwidth}{-.65em}%
      \begin{minipage}[t]{\boxwidth}\footnotesize
      \parskip=0pt\relax}%
       {\end{minipage}\end{lrbox}%
       \settodepth{\alignwidth}{\usebox{\alignbox}}%
      \global\setlength{\multicommentdepth}{\alignwidth}%
      \global\addtolength{\alignwidth}{-\maxdepth}%
      \raisebox{0pt}[0pt][0pt]{%
        \ifthenelse{\boolean{shading}}%
          {\shadebox{\usebox{\alignbox}}}%
          {\usebox{\alignbox}}}%
       \vspace*{\alignwidth}\pagebreak[0]\vspace{-\alignwidth}\par}
 % a multi-line comment, whose first argument is its width in points.


% \multispace{n} produces the vertical space indicated by "|"s in 
% this situation
%   
%     xxxx (*************)
%     xxxx (* ccccccccc *)
%      |   (* ccccccccc *)
%      |   (* ccccccccc *)
%      |   (* ccccccccc *)
%      |   (*************)
%
% where n is the number of "xxxx" lines.
\newcommand{\multivspace}[1]{\addtolength{\multicommentdepth}{-#1\baselineskip}%
 \addtolength{\multicommentdepth}{1.2em}%
 \ifthenelse{\lengthtest{\multicommentdepth > 0pt}}%
    {\par\vspace{\multicommentdepth}\par}{}}

%\newenvironment{hpar}[2]{%
%  \begin{list}{}{\setlength{\leftmargin}{#1pt}%
%                 \addtolength{\leftmargin}{#2pt}%
%                 \setlength{\itemindent}{-#2pt}%
%                 \setlength{\topsep}{0pt}%
%                 \setlength{\parsep}{0pt}%
%                 \setlength{\partopsep}{0pt}%
%                 \setlength{\parskip}{0pt}%
%                 \addtolength{\labelsep}{0pt}}%
%  \item[]\footnotesize}{\end{list}}
%    %%%%%%%%%%%%%%%%%%%%%%%%%%%%%%%%%%%%%%%%%%%%%%%%%%%%%%%%%%%%%%%%%%%%%%%%
%    % Typesets a sequence of paragraphs like this:                         %
%    %                                                                      %
%    %      left |<-- d1 --> XXXXXXXXXXXXXXXXXXXXXXXX                       %
%    %    margin |           <- d2 -> XXXXXXXXXXXXXXX                       %
%    %           |                    XXXXXXXXXXXXXXX                       %
%    %           |                                                          %
%    %           |                    XXXXXXXXXXXXXXX                       %
%    %           |                    XXXXXXXXXXXXXXX                       %
%    %                                                                      %
%    % where d1 = #1pt and d2 = #2pt, but with no vspace between            %
%    % paragraphs.                                                          %
%    %%%%%%%%%%%%%%%%%%%%%%%%%%%%%%%%%%%%%%%%%%%%%%%%%%%%%%%%%%%%%%%%%%%%%%%%

%%%%%%%%%%%%%%%%%%%%%%%%%%%%%%%%%%%%%%%%%%%%%%%%%%%%%%%%%%%%%%%%%%%%%%
% Commands for repeated characters that produce dashes.              %
%%%%%%%%%%%%%%%%%%%%%%%%%%%%%%%%%%%%%%%%%%%%%%%%%%%%%%%%%%%%%%%%%%%%%%
% \raisedDash{wd}{ht}{thk} makes a horizontal line wd characters wide, 
% raised a distance ht ex's above the baseline, with a thickness of 
% thk em's.
\newcommand{\raisedDash}[3]{\raisebox{#2ex}{\setlength{\alignwidth}{.5em}%
  \rule{#1\alignwidth}{#3em}}}

% The following commands take a single argument n and produce the
% output for n repeated characters, as follows
%   \cdash:    -
%   \tdash:    ~
%   \ceqdash:  =
%   \usdash:   _
\newcommand{\cdash}[1]{\raisedDash{#1}{.5}{.04}}
\newcommand{\usdash}[1]{\raisedDash{#1}{0}{.04}}
\newcommand{\ceqdash}[1]{\raisedDash{#1}{.5}{.08}}
\newcommand{\tdash}[1]{\raisedDash{#1}{1}{.08}}

\newlength{\spacewidth}
\setlength{\spacewidth}{.2em}
\newcommand{\e}[1]{\hspace{#1\spacewidth}}
%% \e{i} produces space corresponding to i input spaces.


%% Alignment-file Commands

\newlength{\alignboxwidth}
\newlength{\alignwidth}
\newsavebox{\alignbox}

% \al{i}{j}{txt} is used in the alignment file to put "%{i}{j}{wd}"
% in the log file, where wd is the width of the line up to that point,
% and txt is the following text.
\newcommand{\al}[3]{%
  \typeout{\%{#1}{#2}{\the\alignwidth}}%
  \cl{#3}}

%% \cl{txt} continues a specification line in the alignment file
%% with text txt.
\newcommand{\cl}[1]{%
  \savebox{\alignbox}{\mbox{$\mbox{}#1\mbox{}$}}%
  \settowidth{\alignboxwidth}{\usebox{\alignbox}}%
  \addtolength{\alignwidth}{\alignboxwidth}%
  \usebox{\alignbox}}

% \fl{txt} in the alignment file begins a specification line that
% starts with the text txt.
\newcommand{\fl}[1]{%
  \par
  \savebox{\alignbox}{\mbox{$\mbox{}#1\mbox{}$}}%
  \settowidth{\alignwidth}{\usebox{\alignbox}}%
  \usebox{\alignbox}}



  
%%%%%%%%%%%%%%%%%%%%%%%%%%%%%%%%%%%%%%%%%%%%%%%%%%%%%%%%%%%%%%%%%%%%%%%%%%%%%
% Ordinarily, TeX typesets letters in math mode in a special math italic    %
% font.  This makes it typeset "it" to look like the product of the         %
% variables i and t, rather than like the word "it".  The following         %
% commands tell TeX to use an ordinary italic font instead.                 %
%%%%%%%%%%%%%%%%%%%%%%%%%%%%%%%%%%%%%%%%%%%%%%%%%%%%%%%%%%%%%%%%%%%%%%%%%%%%%
\ifx\documentclass\undefined
\else
  \DeclareSymbolFont{tlaitalics}{\encodingdefault}{cmr}{m}{it}
  \let\itfam\symtlaitalics
\fi

\makeatletter
\newcommand{\tlx@c}{\c@tlx@ctr\advance\c@tlx@ctr\@ne}
\newcounter{tlx@ctr}
\c@tlx@ctr=\itfam \multiply\c@tlx@ctr"100\relax \advance\c@tlx@ctr "7061\relax
\mathcode`a=\tlx@c \mathcode`b=\tlx@c \mathcode`c=\tlx@c \mathcode`d=\tlx@c
\mathcode`e=\tlx@c \mathcode`f=\tlx@c \mathcode`g=\tlx@c \mathcode`h=\tlx@c
\mathcode`i=\tlx@c \mathcode`j=\tlx@c \mathcode`k=\tlx@c \mathcode`l=\tlx@c
\mathcode`m=\tlx@c \mathcode`n=\tlx@c \mathcode`o=\tlx@c \mathcode`p=\tlx@c
\mathcode`q=\tlx@c \mathcode`r=\tlx@c \mathcode`s=\tlx@c \mathcode`t=\tlx@c
\mathcode`u=\tlx@c \mathcode`v=\tlx@c \mathcode`w=\tlx@c \mathcode`x=\tlx@c
\mathcode`y=\tlx@c \mathcode`z=\tlx@c
\c@tlx@ctr=\itfam \multiply\c@tlx@ctr"100\relax \advance\c@tlx@ctr "7041\relax
\mathcode`A=\tlx@c \mathcode`B=\tlx@c \mathcode`C=\tlx@c \mathcode`D=\tlx@c
\mathcode`E=\tlx@c \mathcode`F=\tlx@c \mathcode`G=\tlx@c \mathcode`H=\tlx@c
\mathcode`I=\tlx@c \mathcode`J=\tlx@c \mathcode`K=\tlx@c \mathcode`L=\tlx@c
\mathcode`M=\tlx@c \mathcode`N=\tlx@c \mathcode`O=\tlx@c \mathcode`P=\tlx@c
\mathcode`Q=\tlx@c \mathcode`R=\tlx@c \mathcode`S=\tlx@c \mathcode`T=\tlx@c
\mathcode`U=\tlx@c \mathcode`V=\tlx@c \mathcode`W=\tlx@c \mathcode`X=\tlx@c
\mathcode`Y=\tlx@c \mathcode`Z=\tlx@c
\makeatother

%%%%%%%%%%%%%%%%%%%%%%%%%%%%%%%%%%%%%%%%%%%%%%%%%%%%%%%%%%
%                THE describe ENVIRONMENT                %
%%%%%%%%%%%%%%%%%%%%%%%%%%%%%%%%%%%%%%%%%%%%%%%%%%%%%%%%%%
%
%
% It is like the description environment except it takes an argument
% ARG that should be the text of the widest label.  It adjusts the
% indentation so each item with label LABEL produces
%%      LABEL             blah blah blah
%%      <- width of ARG ->blah blah blah
%%                        blah blah blah
\newenvironment{describe}[1]%
   {\begin{list}{}{\settowidth{\labelwidth}{#1}%
            \setlength{\labelsep}{.5em}%
            \setlength{\leftmargin}{\labelwidth}% 
            \addtolength{\leftmargin}{\labelsep}%
            \addtolength{\leftmargin}{\parindent}%
            \def\makelabel##1{\rm ##1\hfill}}%
            \setlength{\topsep}{0pt}}%% 
                % Sets \topsep to 0 to reduce vertical space above
                % and below embedded displayed equations
   {\end{list}}

%   For tlatex.TeX
\usepackage{verbatim}
\makeatletter
\def\tla{\let\%\relax%
         \@bsphack
         \typeout{\%{\the\linewidth}}%
             \let\do\@makeother\dospecials\catcode`\^^M\active
             \let\verbatim@startline\relax
             \let\verbatim@addtoline\@gobble
             \let\verbatim@processline\relax
             \let\verbatim@finish\relax
             \verbatim@}
\let\endtla=\@esphack


% The tlatex environment is used by TLATeX.TeX to typeset TLA+.
% TLATeX.TLA starts its files by writing a \tlatex command.  This
% command/environment sets \parindent to 0 and defines \% to its
% standard definition because the writing of the log files is messed up
% if \% is defined to be something else.  It also executes
% \@computerule to determine the dimensions for the TLA horizonatl
% bars.
\newenvironment{tlatex}{\@computerule%
                        \setlength{\parindent}{0pt}%
                       \makeatletter\chardef\%=`\%}{}


% The notla environment produces no output.  You can turn a 
% tla environment to a notla environment to prevent tlatex.TeX from
% re-formatting the environment.

\def\notla{\let\%\relax%
         \@bsphack
             \let\do\@makeother\dospecials\catcode`\^^M\active
             \let\verbatim@startline\relax
             \let\verbatim@addtoline\@gobble
             \let\verbatim@processline\relax
             \let\verbatim@finish\relax
             \verbatim@}
\let\endnotla=\@esphack


%%%%%%%%%%%%%%%%%%%%%%%% end of tlatex.sty file %%%%%%%%%%%%%%%%%%%%%%% 
% last modified on Sat 22 Sep 2007 at  8:45:06 PST by lamport

\begin{document}
\tlatex
\setboolean{shading}{true}
\@x{}\moduleLeftDash\@xx{ {\MODULE} NBlockGraph}\moduleRightDash\@xx{}%
\begin{lcom}{0}%
\begin{cpar}{0}{F}{F}{0}{0}{}%
The \ensuremath{NBlockGraph} module defines a data structure which allows
 processors to acquire \ensuremath{NBlocks} in such a way that no two
 processors
 ever have \ensuremath{NBlocks} that are in each other\mbox{'}s protection
 scopes. The
 actions in this module model the \ensuremath{next\_nblock()} function in
 \ensuremath{nblock\_graph.cc}.
\end{cpar}%
\end{lcom}%
\@x{ {\EXTENDS} FiniteSets ,\, Naturals ,\, Lock}%
\par\vspace{16.0pt}%
\@x{ {\CONSTANTS}}%
\@x{\@s{32.8} NNBlocks ,\,\@s{24.54}}%
\@y{\@s{0}%
 The number of \ensuremath{NBlocks}.
}%
\@xx{}%
\@x{\@s{32.8} NProcs ,\,\@s{35.68}}%
\@y{\@s{0}%
 The number of processors.
}%
\@xx{}%
\@x{\@s{32.8} None ,\,\@s{44.20}}%
\@y{\@s{0}%
 No \ensuremath{NBlock}.
}%
\@xx{}%
\@x{\@s{32.8} Search ,\,\@s{38.71}}%
\@y{\@s{0}%
 Process will search next.
}%
\@xx{}%
\@x{\@s{32.8} NewBlock ,\,\@s{24.57}}%
\@y{\@s{0}%
 Process will get a new \ensuremath{NBlock}.
}%
\@xx{}%
\@x{\@s{32.8} GetBlock ,\,\@s{27.59}}%
\@y{\@s{0}%
 Process tries to get the next free block.
}%
\@xx{}%
\@x{\@s{32.8} Waiting ,\,\@s{32.05}}%
\@y{\@s{0}%
 Process wakes up after failing to get a free block.
}%
\@xx{}%
\@x{\@s{32.8} PredsOf ( \_ ,\, \_ ) ,\,\@s{8.19}}%
\@y{\@s{0}%
 Predecessors of the given \ensuremath{NBlock
}}%
\@xx{}%
\@x{\@s{32.8} SuccsOf ( \_ ,\, \_ )\@s{14.32}}%
\@y{\@s{0}%
 Successors of the given \ensuremath{NBlock
}}%
\@xx{}%
\par\vspace{16.0pt}%
\begin{lcom}{0}%
\begin{cpar}{0}{F}{F}{0}{0}{}%
All of the possible states that a process can be in.
\end{cpar}%
\end{lcom}%
\@x{ States \.{\defeq} \{ Search ,\, NewBlock ,\, GetBlock ,\, Waiting \}}%
\par\vspace{16.0pt}%
\@x{ {\VARIABLES}}%
\@x{\@s{32.8} state ,\,\@s{43.67}}%
\@y{\@s{0}%
 The state of each processe.
}%
\@xx{}%
\@x{\@s{32.8} acquired ,\,\@s{28.70}}%
\@y{\@s{0}%
 The \ensuremath{NBlock} that each process has acquired.
}%
\@xx{}%
\@x{\@s{32.8} hot ,\,\@s{50.37}}%
\@y{\@s{0}%
 Set when an \ensuremath{NBlock} becomes hot.
}%
\@xx{}%
\@x{\@s{32.8} waiting ,\,\@s{32.55}}%
\@y{\@s{0}%
 Set of processes waiting for a free \ensuremath{NBlock}.
}%
\@xx{}%
\@x{\@s{32.8} lock\@s{53.55}}%
\@y{\@s{0}%
 The \ensuremath{PID} of the process with the lock.
}%
\@xx{}%
\par\vspace{16.0pt}%
\begin{lcom}{0}%
\begin{cpar}{0}{F}{F}{0}{0}{}%
The set of all of the \ensuremath{NBlocks}.
\end{cpar}%
\end{lcom}%
\@x{ NBlocks \.{\defeq} 0 \.{\dotdot} NNBlocks \.{-} 1}%
\par\vspace{16.0pt}%
\begin{lcom}{0}%
\begin{cpar}{0}{F}{F}{0}{0}{}%
The predecessors of the given \ensuremath{NBlocks}.
\end{cpar}%
\end{lcom}%
\@x{ Preds ( b ) \.{\defeq} PredsOf ( NBlocks ,\, b )}%
\par\vspace{16.0pt}%
\begin{lcom}{0}%
\begin{cpar}{0}{F}{F}{0}{0}{}%
The successors of the given \ensuremath{NBlocks}.
\end{cpar}%
\end{lcom}%
\@x{ Succs ( b ) \.{\defeq} SuccsOf ( NBlocks ,\, b )}%
\par\vspace{16.0pt}%
\begin{lcom}{0}%
\begin{cpar}{0}{F}{F}{0}{0}{}%
Set of all process \ensuremath{IDs
}%
\end{cpar}%
\end{lcom}%
\@x{ ProcIds \.{\defeq} 0 \.{\dotdot} NProcs \.{-} 1}%
\par\vspace{16.0pt}%
\begin{lcom}{0}%
\begin{cpar}{0}{F}{F}{0}{0}{}%
A list of all of the variables.
\end{cpar}%
\end{lcom}%
 \@x{ Vars \.{\defeq} {\langle} state ,\, acquired ,\, hot ,\, waiting ,\,
 lock {\rangle}}%
\par\vspace{16.0pt}%
\@x{ {\ASSUME}\@s{4.1} \.{\land} NNBlocks \.{\geq} NProcs}%
\@x{\@s{42.34} \.{\land} None \.{\notin} ( NBlocks \.{\cup} NProcs )}%
\@x{\@s{42.34} \.{\land} Cardinality ( States ) \.{=} 4}%
\par\vspace{16.0pt}%
\begin{lcom}{0}%
\begin{cpar}{0}{F}{F}{0}{0}{}%
The types of all of the variables. These should always remain
 true.
\end{cpar}%
\end{lcom}%
 \@x{ TypeInv \.{\defeq} \.{\land} state \.{\in} [ ProcIds \.{\rightarrow}
 States ]}%
 \@x{\@s{54.48} \.{\land} acquired \.{\in} [ ProcIds \.{\rightarrow} NBlocks
 \.{\cup} \{ None \} ]}%
\@x{\@s{54.48} \.{\land} hot \.{\in} [ NBlocks \.{\rightarrow} {\BOOLEAN} ]}%
\@x{\@s{54.48} \.{\land} waiting \.{\subseteq} ProcIds}%
\@x{\@s{54.48} \.{\land} lock \.{\in} LockType ( ProcIds )}%
\@x{\@s{54.48} \.{\land} LockInvariant ( lock )}%
\par\vspace{8.0pt}%
\begin{lcom}{0}%
\begin{cpar}{0}{F}{F}{0}{0}{}%
The initial state of the system.
\end{cpar}%
\end{lcom}%
 \@x{ Init \.{\defeq} \.{\land} state \.{=} [ x \.{\in} ProcIds \.{\mapsto}
 NewBlock ]}%
 \@x{\@s{35.70} \.{\land} acquired\@s{0.27} \.{=} [ x \.{\in} ProcIds
 \.{\mapsto} None ]}%
 \@x{\@s{35.70} \.{\land} hot \.{=} [ x \.{\in} NBlocks \.{\mapsto} {\FALSE}
 ]}%
\@x{\@s{35.70} \.{\land} waiting \.{=} \{ \}}%
\@x{\@s{35.70} \.{\land} lock \.{=} Unlocked}%
\@x{\@s{35.70} \.{\land} TypeInv}%
\par\vspace{16.0pt}%
\begin{lcom}{0}%
\begin{cpar}{0}{F}{F}{0}{0}{}%
 Get all of the \ensuremath{NBlocks} that the \ensuremath{NBlock}, given by
 \ensuremath{y}, interferes
 with.
\end{cpar}%
\end{lcom}%
 \@x{ InterferenceScope ( y ) \.{\defeq} \.{\LET} SuccPreds \.{\defeq} \{ x
 \.{\in} NBlocks \.{:} \E\, z \.{\in} Succs ( y ) \.{:} x \.{\in} Preds ( z )
 \.{\land} x \.{\neq} y \}}%
 \@x{\@s{109.07} \.{\IN} \{ x \.{\in} NBlocks \.{:} \.{\lor} ( x \.{\in} Preds
 ( y ) \.{\land} x \.{\neq} y )}%
\@x{\@s{197.72} \.{\lor} ( x \.{\in} SuccPreds \.{\land} x \.{\neq} y ) \}}%
\par\vspace{16.0pt}%
\begin{lcom}{0}%
\begin{cpar}{0}{F}{F}{0}{0}{}%
The sigma value for an \ensuremath{NBlock} is the number of in-use successors.
\end{cpar}%
\end{lcom}%
 \@x{ Sigma ( b ) \.{\defeq} Cardinality ( \{ x \.{\in} Succs ( b ) \.{:} \E\,
 i \.{\in} ProcIds \.{:} \.{\lor} acquired [ i ] \.{=} x}%
 \@x{\@s{249.34} \.{\lor} \E\, j \.{\in} Preds ( x ) \.{\,\backslash\,} \{ b
 \} \.{:} acquired [ i ] \.{=} j \} )}%
\par\vspace{16.0pt}%
\begin{lcom}{0}%
\begin{cpar}{0}{F}{F}{0}{0}{}%
 The \ensuremath{sigma\_hot} value is the number of \ensuremath{NBlocks} in
 the interference
 scope that are hot.
\end{cpar}%
\end{lcom}%
 \@x{ Sigma\_hot ( b ) \.{\defeq} Cardinality ( \{ x \.{\in} InterferenceScope
 ( b ) \.{:} hot [ x ] \} )}%
\par\vspace{16.0pt}%
\begin{lcom}{0}%
\begin{cpar}{0}{F}{F}{0}{0}{}%
 The \ensuremath{FreeList} is all \ensuremath{NBlocks} that have sigma 0 and
 aren\mbox{'}t acquired.
\end{cpar}%
\vshade{10.0}%
\begin{cpar}{0}{F}{F}{0}{0}{}%
 \ensuremath{\.{\,\backslash\,}}* This \ensuremath{FreeList} will demonstrate
 the live lock.
\end{cpar}%
\begin{cpar}{0}{F}{F}{0}{0}{}%
 \ensuremath{FreeList \.{\defeq} \{x \.{\in} NBlocks \.{:} \.{\land} Sigma(x)
 \.{=} 0
}%
\end{cpar}%
\begin{cpar}{0}{T}{F}{72.5}{0}{}%
\ensuremath{\.{\land} \A\, y \.{\in} ProcIds \.{:} acquired[y] \.{\neq} x\}
}%
\end{cpar}%
\end{lcom}%
\@x{}%
\@y{\@s{0}%
 This \ensuremath{FreeList} will fix the live lock.
}%
\@xx{}%
 \@x{ FreeList \.{\defeq} \{ x \.{\in} NBlocks \.{:} \.{\land} Sigma ( x )
 \.{=} 0}%
\@x{\@s{122.98} \.{\land} Sigma\_hot ( x ) \.{=} 0}%
 \@x{\@s{122.98} \.{\land} \A\, y \.{\in} ProcIds \.{:} acquired [ y ]
 \.{\neq} x \}}%
\par\vspace{16.0pt}%
\@x{}\midbar\@xx{}%
\par\vspace{16.0pt}%
\begin{lcom}{0}%
\begin{cpar}{0}{F}{F}{0}{0}{}%
A process is getting a new \ensuremath{NBlock}, and releasing their old one.
\end{cpar}%
\end{lcom}%
\@x{ DoNewBlock ( i ) \.{\defeq} \.{\land} state [ i ] \.{=} NewBlock}%
\@x{\@s{86.56} \.{\land} {\IF} acquired [ i ] \.{=} None}%
\@x{\@s{97.67}}%
\@y{\@s{0}%
 If we didn\mbox{'}t have a block, there is nothing to release.
}%
\@xx{}%
 \@x{\@s{97.67} \.{\THEN} \.{\land} state \.{'} \.{=} [ state {\EXCEPT}
 {\bang} [ i ] \.{=} GetBlock ]}%
\@x{\@s{128.98} \.{\land} Lock ( lock ,\, i )}%
 \@x{\@s{128.98} \.{\land} {\UNCHANGED} {\langle} acquired ,\, hot ,\, waiting
 {\rangle}}%
\@x{\@s{97.67}}%
\@y{\@s{0}%
 Release the acquired block and the get another.
}%
\@xx{}%
 \@x{\@s{97.67} \.{\ELSE} \.{\land} state \.{'} \.{=} [ state {\EXCEPT}
 {\bang} [ i ] \.{=} GetBlock ]}%
\@x{\@s{128.98} \.{\land} Lock ( lock ,\, i )}%
 \@x{\@s{128.98} \.{\land} acquired \.{'}\@s{0.27} \.{=} [ acquired {\EXCEPT}
 {\bang} [ i ] \.{=} None ]}%
 \@x{\@s{128.98} \.{\land} hot \.{'} \.{=} [ x \.{\in} NBlocks \.{\mapsto}
 {\IF} Sigma ( x ) \.{>} 0 \.{\land} Sigma ( x ) \.{'} \.{=} 0}%
\@x{\@s{243.97} \.{\THEN} {\FALSE}}%
\@x{\@s{243.97} \.{\ELSE} hot [ x ] ]}%
 \@x{\@s{128.98} \.{\land} {\IF} \E\, x \.{\in} NBlocks \.{:} Sigma ( x )
 \.{>} 0 \.{\land} Sigma ( x ) \.{'} \.{=} 0}%
\@x{\@s{140.09} \.{\THEN} waiting \.{'} \.{=} \{ \}}%
\@x{\@s{140.09} \.{\ELSE} {\UNCHANGED} {\langle} waiting {\rangle}}%
\par\vspace{16.0pt}%
\begin{lcom}{0}%
\begin{cpar}{0}{F}{F}{0}{0}{}%
Get the next block.
\end{cpar}%
\end{lcom}%
\@x{ DoGetBlock ( i ) \.{\defeq} \.{\land} state [ i ] \.{=} GetBlock}%
\@x{\@s{83.54} \.{\land} HasLock ( lock ,\, i )}%
\@x{\@s{83.54} \.{\land} {\IF} FreeList \.{=} \{ \}}%
 \@x{\@s{94.65} \.{\THEN} \.{\land} state \.{'} \.{=} [ state {\EXCEPT}
 {\bang} [ i ] \.{=} Waiting ]}%
\@x{\@s{125.96} \.{\land} Unlock ( lock )}%
\@x{\@s{125.96} \.{\land} waiting \.{'} \.{=} waiting \.{\cup} \{ i \}}%
\@x{\@s{125.96} \.{\land} {\UNCHANGED} {\langle} acquired {\rangle}}%
 \@x{\@s{94.65} \.{\ELSE} \.{\land} state \.{'} \.{=} [ state {\EXCEPT}
 {\bang} [ i ] \.{=} Search ]}%
 \@x{\@s{125.96} \.{\land} \E\, b \.{\in} FreeList \.{:} acquired \.{'} \.{=}
 [ acquired {\EXCEPT} {\bang} [ i ] \.{=} b ]}%
\@x{\@s{125.96} \.{\land} Unlock ( lock )}%
\@x{\@s{125.96} \.{\land} {\UNCHANGED} {\langle} waiting {\rangle}}%
\@x{\@s{83.54} \.{\land} {\UNCHANGED} {\langle} hot {\rangle}}%
\par\vspace{16.0pt}%
\begin{lcom}{0}%
\begin{cpar}{0}{F}{F}{0}{0}{}%
Wakeup after waiting, re-lock and test the loop condition again.
\end{cpar}%
\end{lcom}%
\@x{ DoWaiting ( i ) \.{\defeq} \.{\land} state [ i ] \.{=} Waiting}%
\@x{\@s{79.07} \.{\land} Lock ( lock ,\, i )}%
 \@x{\@s{79.07} \.{\land} state \.{'} \.{=} [ state {\EXCEPT} {\bang} [ i ]
 \.{=} GetBlock ]}%
 \@x{\@s{79.07} \.{\land} {\UNCHANGED} {\langle} acquired ,\, hot ,\, waiting
 {\rangle}}%
\par\vspace{16.0pt}%
\begin{lcom}{0}%
\begin{cpar}{0}{F}{F}{0}{0}{}%
Search the acquired \ensuremath{NBlock}.
\end{cpar}%
\end{lcom}%
\@x{ DoSearch ( i ) \.{\defeq} \.{\land} state [ i ] \.{=} Search}%
\@x{\@s{72.41} \.{\land} acquired [ i ] \.{\neq} None}%
\@x{\@s{72.41} \.{\land}}%
\@y{\@s{0}%
 Maybe set an interfered block to Hot
}%
\@xx{}%
 \@x{\@s{83.52} \.{\lor} \E\, b \.{\in} InterferenceScope ( acquired [ i ] )
 \.{:} \.{\land} {\lnot} hot [ b ]}%
 \@x{\@s{259.02} \.{\land} {\lnot} ( \E\, j \.{\in} InterferenceScope ( b )
 \.{:} hot [ j ] )}%
 \@x{\@s{259.02} \.{\land} hot \.{'} \.{=} [ hot {\EXCEPT} {\bang} [ b ] \.{=}
 {\TRUE} ]}%
\@x{\@s{83.52} \.{\lor} {\UNCHANGED} {\langle} hot {\rangle}}%
 \@x{\@s{72.41} \.{\land} state \.{'} \.{=} [ state {\EXCEPT} {\bang} [ i ]
 \.{=} NewBlock ]}%
 \@x{\@s{72.41} \.{\land} {\UNCHANGED} {\langle} acquired ,\, waiting ,\, lock
 {\rangle}}%
\par\vspace{16.0pt}%
\@x{ Advance ( i ) \.{\defeq} \.{\land} i \.{\in} ProcIds}%
\@x{\@s{67.55} \.{\land} i \.{\notin} waiting}%
\@x{\@s{67.55} \.{\land} \.{\lor} DoNewBlock ( i )}%
\@x{\@s{78.66} \.{\lor} DoGetBlock ( i )}%
\@x{\@s{78.66} \.{\lor} DoWaiting ( i )}%
\@x{\@s{78.66} \.{\lor} DoSearch ( i )}%
\par\vspace{16.0pt}%
\@x{ Next \.{\defeq} \E\, i \.{\in} ProcIds \.{:} Advance ( i )}%
\par\vspace{16.0pt}%
 \@x{ Fairness \.{\defeq} \A\, i \.{\in} ProcIds \.{:} {\WF}_{ {\langle} state
 ,\, acquired ,\, hot ,\, waiting ,\, lock {\rangle}} ( Advance ( i ) )}%
\par\vspace{8.0pt}%
\@x{}\midbar\@xx{}%
\par\vspace{8.0pt}%
\begin{lcom}{0}%
\begin{cpar}{0}{F}{F}{0}{0}{}%
An \ensuremath{NBlock} is never acquired by more than one processor.
\end{cpar}%
\end{lcom}%
 \@x{ DifferentNBlocks \.{\defeq} {\Box} ( \A\, i \.{\in} ProcIds \.{:}
 acquired [ i ] \.{\neq} None \.{\implies} {\lnot} ( \E\, j \.{\in} ProcIds
 \.{\,\backslash\,} \{ i \} \.{:} acquired [ j ] \.{=} acquired [ i ] ) )}%
\par\vspace{16.0pt}%
\begin{lcom}{0}%
\begin{cpar}{0}{F}{F}{0}{0}{}%
If a processor has acquired an \ensuremath{NBlock}, then none of its the
 \ensuremath{NBlock}\mbox{'}s \ensuremath{Succs} are acquired at the same
 time.
\end{cpar}%
\end{lcom}%
\@x{ NoInterference \.{\defeq} {\Box} ( \A\, i \.{\in} ProcIds \.{:}}%
 \@x{\@s{99.54} acquired [ i ] \.{\neq} None \.{\implies} {\lnot} ( \E\, j
 \.{\in} ProcIds \.{\,\backslash\,} \{ i \} \.{:} acquired [ j ] \.{\in}
 InterferenceScope ( acquired [ i ] ) ) )}%
\par\vspace{16.0pt}%
\begin{lcom}{0}%
\begin{cpar}{0}{F}{F}{0}{0}{}%
 If a process has acquired an \ensuremath{NBlock}, than than block\mbox{'}s
 sigma value
 is zero.
\end{cpar}%
\end{lcom}%
 \@x{ SigmaZeroIfAcquired \.{\defeq} {\Box} ( \A\, i \.{\in} ProcIds \.{:}
 acquired [ i ] \.{\neq} None \.{\implies} Sigma ( acquired [ i ] ) \.{=} 0
 )}%
\par\vspace{16.0pt}%
\begin{lcom}{0}%
\begin{cpar}{0}{F}{F}{0}{0}{}%
If an \ensuremath{NBlock} has a non-zero sigma value, then it is not on the
 \ensuremath{FreeList}, and no process has acquired it.
\end{cpar}%
\end{lcom}%
 \@x{ SigmaNonZero \.{\defeq} {\Box} ( \A\, i \.{\in} NBlocks \.{:} Sigma ( i
 ) \.{\neq} 0 \.{\implies} ( i \.{\notin} FreeList \.{\land} {\lnot} ( \E\, j
 \.{\in} ProcIds \.{:} acquired [ j ] \.{=} i ) ) )}%
\par\vspace{16.0pt}%
\begin{lcom}{0}%
\begin{cpar}{0}{F}{F}{0}{0}{}%
It is never the case that all of the processors are waiting.
\end{cpar}%
\end{lcom}%
 \@x{ NoDeadlock \.{\defeq} {\Box} ( \E\, i \.{\in} ProcIds \.{:} i \.{\notin}
 waiting )}%
\par\vspace{16.0pt}%
\begin{lcom}{0}%
\begin{cpar}{0}{F}{F}{0}{0}{}%
If a block becomes hot, it will eventually become cold.
\end{cpar}%
\end{lcom}%
 \@x{ ChillOut \.{\defeq} \A\, i \.{\in} NBlocks \.{:} hot [ i ] \.{\leadsto}
 ( \E\, j \.{\in} ProcIds \.{:} acquired [ j ] \.{=} i \.{\lor} i \.{\in}
 FreeList )}%
\par\vspace{16.0pt}%
\begin{lcom}{0}%
\begin{cpar}{0}{F}{F}{0}{0}{}%
If a process tries to get the lock, they will eventually get it.
\end{cpar}%
\end{lcom}%
 \@x{ NoLockStarvation \.{\defeq} {\Box} ( \A\, i \.{\in} ProcIds \.{:}
 IsWaitingOnLock ( lock ,\, i ) \.{\leadsto} HasLock ( lock ,\, i ) )}%
\par\vspace{16.0pt}%
\begin{lcom}{0}%
\begin{cpar}{0}{F}{F}{0}{0}{}%
There is never more than one process that gets the lock.
\end{cpar}%
\end{lcom}%
 \@x{ MutualExclusion \.{\defeq} {\Box} ( \A\, i \.{\in} ProcIds \.{:} HasLock
 ( lock ,\, i ) \.{\implies} \E\, j \.{\in} ProcIds \.{:} HasLock ( lock ,\,
 j ) \.{\implies} j \.{=} i )}%
\par\vspace{16.0pt}%
\begin{lcom}{0}%
\begin{cpar}{0}{F}{F}{0}{0}{}%
If a process gets the lock, they will eventually release it.
\end{cpar}%
\end{lcom}%
 \@x{ EventuallyReleased \.{\defeq} {\Box} ( \A\, i \.{\in} ProcIds \.{:}
 HasLock ( lock ,\, i ) \.{\leadsto} {\lnot} HasLock ( lock ,\, i ) )}%
\par\vspace{8.0pt}%
\@x{}\bottombar\@xx{}%
\end{document}
